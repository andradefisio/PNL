% Options for packages loaded elsewhere
\PassOptionsToPackage{unicode}{hyperref}
\PassOptionsToPackage{hyphens}{url}
%
\documentclass[
]{article}
\usepackage{lmodern}
\usepackage{amssymb,amsmath}
\usepackage{ifxetex,ifluatex}
\ifnum 0\ifxetex 1\fi\ifluatex 1\fi=0 % if pdftex
  \usepackage[T1]{fontenc}
  \usepackage[utf8]{inputenc}
  \usepackage{textcomp} % provide euro and other symbols
\else % if luatex or xetex
  \usepackage{unicode-math}
  \defaultfontfeatures{Scale=MatchLowercase}
  \defaultfontfeatures[\rmfamily]{Ligatures=TeX,Scale=1}
\fi
% Use upquote if available, for straight quotes in verbatim environments
\IfFileExists{upquote.sty}{\usepackage{upquote}}{}
\IfFileExists{microtype.sty}{% use microtype if available
  \usepackage[]{microtype}
  \UseMicrotypeSet[protrusion]{basicmath} % disable protrusion for tt fonts
}{}
\makeatletter
\@ifundefined{KOMAClassName}{% if non-KOMA class
  \IfFileExists{parskip.sty}{%
    \usepackage{parskip}
  }{% else
    \setlength{\parindent}{0pt}
    \setlength{\parskip}{6pt plus 2pt minus 1pt}}
}{% if KOMA class
  \KOMAoptions{parskip=half}}
\makeatother
\usepackage{xcolor}
\IfFileExists{xurl.sty}{\usepackage{xurl}}{} % add URL line breaks if available
\IfFileExists{bookmark.sty}{\usepackage{bookmark}}{\usepackage{hyperref}}
\hypersetup{
  pdftitle={Tidytext\_estudo},
  pdfauthor={Antonio\_Andrade},
  hidelinks,
  pdfcreator={LaTeX via pandoc}}
\urlstyle{same} % disable monospaced font for URLs
\usepackage[margin=1in]{geometry}
\usepackage{color}
\usepackage{fancyvrb}
\newcommand{\VerbBar}{|}
\newcommand{\VERB}{\Verb[commandchars=\\\{\}]}
\DefineVerbatimEnvironment{Highlighting}{Verbatim}{commandchars=\\\{\}}
% Add ',fontsize=\small' for more characters per line
\usepackage{framed}
\definecolor{shadecolor}{RGB}{248,248,248}
\newenvironment{Shaded}{\begin{snugshade}}{\end{snugshade}}
\newcommand{\AlertTok}[1]{\textcolor[rgb]{0.94,0.16,0.16}{#1}}
\newcommand{\AnnotationTok}[1]{\textcolor[rgb]{0.56,0.35,0.01}{\textbf{\textit{#1}}}}
\newcommand{\AttributeTok}[1]{\textcolor[rgb]{0.77,0.63,0.00}{#1}}
\newcommand{\BaseNTok}[1]{\textcolor[rgb]{0.00,0.00,0.81}{#1}}
\newcommand{\BuiltInTok}[1]{#1}
\newcommand{\CharTok}[1]{\textcolor[rgb]{0.31,0.60,0.02}{#1}}
\newcommand{\CommentTok}[1]{\textcolor[rgb]{0.56,0.35,0.01}{\textit{#1}}}
\newcommand{\CommentVarTok}[1]{\textcolor[rgb]{0.56,0.35,0.01}{\textbf{\textit{#1}}}}
\newcommand{\ConstantTok}[1]{\textcolor[rgb]{0.00,0.00,0.00}{#1}}
\newcommand{\ControlFlowTok}[1]{\textcolor[rgb]{0.13,0.29,0.53}{\textbf{#1}}}
\newcommand{\DataTypeTok}[1]{\textcolor[rgb]{0.13,0.29,0.53}{#1}}
\newcommand{\DecValTok}[1]{\textcolor[rgb]{0.00,0.00,0.81}{#1}}
\newcommand{\DocumentationTok}[1]{\textcolor[rgb]{0.56,0.35,0.01}{\textbf{\textit{#1}}}}
\newcommand{\ErrorTok}[1]{\textcolor[rgb]{0.64,0.00,0.00}{\textbf{#1}}}
\newcommand{\ExtensionTok}[1]{#1}
\newcommand{\FloatTok}[1]{\textcolor[rgb]{0.00,0.00,0.81}{#1}}
\newcommand{\FunctionTok}[1]{\textcolor[rgb]{0.00,0.00,0.00}{#1}}
\newcommand{\ImportTok}[1]{#1}
\newcommand{\InformationTok}[1]{\textcolor[rgb]{0.56,0.35,0.01}{\textbf{\textit{#1}}}}
\newcommand{\KeywordTok}[1]{\textcolor[rgb]{0.13,0.29,0.53}{\textbf{#1}}}
\newcommand{\NormalTok}[1]{#1}
\newcommand{\OperatorTok}[1]{\textcolor[rgb]{0.81,0.36,0.00}{\textbf{#1}}}
\newcommand{\OtherTok}[1]{\textcolor[rgb]{0.56,0.35,0.01}{#1}}
\newcommand{\PreprocessorTok}[1]{\textcolor[rgb]{0.56,0.35,0.01}{\textit{#1}}}
\newcommand{\RegionMarkerTok}[1]{#1}
\newcommand{\SpecialCharTok}[1]{\textcolor[rgb]{0.00,0.00,0.00}{#1}}
\newcommand{\SpecialStringTok}[1]{\textcolor[rgb]{0.31,0.60,0.02}{#1}}
\newcommand{\StringTok}[1]{\textcolor[rgb]{0.31,0.60,0.02}{#1}}
\newcommand{\VariableTok}[1]{\textcolor[rgb]{0.00,0.00,0.00}{#1}}
\newcommand{\VerbatimStringTok}[1]{\textcolor[rgb]{0.31,0.60,0.02}{#1}}
\newcommand{\WarningTok}[1]{\textcolor[rgb]{0.56,0.35,0.01}{\textbf{\textit{#1}}}}
\usepackage{graphicx,grffile}
\makeatletter
\def\maxwidth{\ifdim\Gin@nat@width>\linewidth\linewidth\else\Gin@nat@width\fi}
\def\maxheight{\ifdim\Gin@nat@height>\textheight\textheight\else\Gin@nat@height\fi}
\makeatother
% Scale images if necessary, so that they will not overflow the page
% margins by default, and it is still possible to overwrite the defaults
% using explicit options in \includegraphics[width, height, ...]{}
\setkeys{Gin}{width=\maxwidth,height=\maxheight,keepaspectratio}
% Set default figure placement to htbp
\makeatletter
\def\fps@figure{htbp}
\makeatother
\setlength{\emergencystretch}{3em} % prevent overfull lines
\providecommand{\tightlist}{%
  \setlength{\itemsep}{0pt}\setlength{\parskip}{0pt}}
\setcounter{secnumdepth}{-\maxdimen} % remove section numbering

\title{Tidytext\_estudo}
\author{Antonio\_Andrade}
\date{13/05/2022}

\begin{document}
\maketitle

\hypertarget{introduuxe7uxe3o}{%
\section{1 Introdução}\label{introduuxe7uxe3o}}

A pandêmia de COVID-19 causou efeitos nefastos à população mundial. Até
a escrita deste ensaio, os dados dão conta de que, no Brasil, foram
ceifadas 666.971 vidas pela COVID-19, e o número acumulado de casos
confirmados atingiu patamares próximos à estratosfera, ou algo na ordem
dos 31,14 milhões
(\url{https://ourworldindata.org/coronavirus/country/brazil} - acesso em
04/06/2022). Se hoje o número de mortes diárias está em 93, no auge ou
pico da pandemia, em 01 de Abril de 2021, foram registradas 3.107 mortes
(\url{https://ourworldindata.org/coronavirus/country/brazil} - acesso em
04/06/2022). Tendo em vista este cenário devastador de proporções
globais, as entidades sanitárias e a ``Wold Health Organization'' (WHO)
propuseram algumas medidas no sentido de conter o avanço da pandemia e
conseguintes complicações entre os indivíduos em geral e nos grupos
vulneráveis (aqueles com comorbidades e idosos). Dentre as sugestões
propostas estão: manter distanciamento físico entre indivíduos de pelo
menos 1 metro de distância; evitar multidões e contato muito próximo;
uso da máscara facial devidamente ajustada quando o distanciamento
físico não for possível e em ambientes mal ventilados; limpar as mãos
frequentemente com álcool gel ou água e sabão; vacinar-se assim que
chegar a sua vez (
\url{https://www.who.int/emergencies/diseases/novel-coronavirus-2019/advice-for-public}
). E é exatamente neste último item pontuado que convergem as discussões
as quais vão suscitar indiretamente o objeto de estudo do ensaio.
Primeiro foram as pontuações referentes à leniência do poder público no
enfrentamento das questões relacionadas à pandemia. A vacina foi
apontada como uma medida eficaz para conter a pandemia. Inicialmente
havia um escassez de vacinas disponíveis no mercado farmacêutico
internacional. Sendo assim, uma demora ou ausência na declaração de
interesse de compra frente a estas multinacionais produtoras da vacina
poderia representar um risco de aquisição de um número menor de doses
necessárias para atender com equidade toda a população brasileira. Outro
ponto levantado foram as questões referentes ao superfaturamento do
preço das doses a serem adquiridas, além da suspeição de desvio de
dinheiro público para interesses particulares.

Antes de enumerá-las, acredito que seja necessário enfatizar que como
escritor deste ensaio, declaro não haver nenhum conflito de interesse
inerente a estas discussões, assegurando para inteiramente manter um
esforço hercúleo de neutralidade, reservando ou concentrando apenas às
questões metodológicas usadas para tentar responder às questões ou
hipóteses formuladas.

Este documento é um sumario de estudos do Processamento de Linguagem
Natural (PLN), baseado no livro de Julie Silge e outras fontes (citar).
Serão feitas algumas simulações no R, para entendimentos da manipulação
de textos. A princípio, a base de dados usada será de um ``wabscraping''
de discursos da ``CPI da Pandemia'', disponivel em
\url{https://basedosdados.org/dataset/br-senado-cpipandemia} .

\begin{Shaded}
\begin{Highlighting}[]
\CommentTok{# Carregando os pacotes necessários:}

\KeywordTok{library}\NormalTok{(tidyverse)}
\end{Highlighting}
\end{Shaded}

\begin{verbatim}
## -- Attaching packages ------------------------------------------------ tidyverse 1.3.0 --
\end{verbatim}

\begin{verbatim}
## v ggplot2 3.3.3     v purrr   0.3.4
## v tibble  3.0.3     v dplyr   1.0.2
## v tidyr   1.1.2     v stringr 1.4.0
## v readr   1.3.1     v forcats 0.5.0
\end{verbatim}

\begin{verbatim}
## Warning: package 'ggplot2' was built under R version 4.0.4
\end{verbatim}

\begin{verbatim}
## -- Conflicts --------------------------------------------------- tidyverse_conflicts() --
## x dplyr::filter() masks stats::filter()
## x dplyr::lag()    masks stats::lag()
\end{verbatim}

\begin{Shaded}
\begin{Highlighting}[]
\KeywordTok{library}\NormalTok{(ggExtra)}
\end{Highlighting}
\end{Shaded}

\begin{verbatim}
## Warning: package 'ggExtra' was built under R version 4.0.5
\end{verbatim}

\begin{Shaded}
\begin{Highlighting}[]
\KeywordTok{library}\NormalTok{(magrittr)}
\end{Highlighting}
\end{Shaded}

\begin{verbatim}
## 
## Attaching package: 'magrittr'
\end{verbatim}

\begin{verbatim}
## The following object is masked from 'package:purrr':
## 
##     set_names
\end{verbatim}

\begin{verbatim}
## The following object is masked from 'package:tidyr':
## 
##     extract
\end{verbatim}

\begin{Shaded}
\begin{Highlighting}[]
\KeywordTok{library}\NormalTok{(lubridate)}
\end{Highlighting}
\end{Shaded}

\begin{verbatim}
## 
## Attaching package: 'lubridate'
\end{verbatim}

\begin{verbatim}
## The following objects are masked from 'package:base':
## 
##     date, intersect, setdiff, union
\end{verbatim}

\begin{Shaded}
\begin{Highlighting}[]
\KeywordTok{library}\NormalTok{(stringr) }
\KeywordTok{library}\NormalTok{(stopwords)      }\CommentTok{#p/palavras "quebra de frase"}
\end{Highlighting}
\end{Shaded}

\begin{verbatim}
## Warning: package 'stopwords' was built under R version 4.0.5
\end{verbatim}

\begin{Shaded}
\begin{Highlighting}[]
\KeywordTok{library}\NormalTok{(tm)             }\CommentTok{#text mining}
\end{Highlighting}
\end{Shaded}

\begin{verbatim}
## Warning: package 'tm' was built under R version 4.0.5
\end{verbatim}

\begin{verbatim}
## Loading required package: NLP
\end{verbatim}

\begin{verbatim}
## Warning: package 'NLP' was built under R version 4.0.3
\end{verbatim}

\begin{verbatim}
## 
## Attaching package: 'NLP'
\end{verbatim}

\begin{verbatim}
## The following object is masked from 'package:ggplot2':
## 
##     annotate
\end{verbatim}

\begin{verbatim}
## 
## Attaching package: 'tm'
\end{verbatim}

\begin{verbatim}
## The following object is masked from 'package:stopwords':
## 
##     stopwords
\end{verbatim}

\begin{Shaded}
\begin{Highlighting}[]
\KeywordTok{library}\NormalTok{(tidytext)}

\KeywordTok{library}\NormalTok{(lexiconPT) }\CommentTok{#lexicon da Língua Portuguesa}
\end{Highlighting}
\end{Shaded}

\begin{verbatim}
## Warning: package 'lexiconPT' was built under R version 4.0.3
\end{verbatim}

\hypertarget{importauxe7uxe3o-dos-arquivos}{%
\subsection{2 Importação doS
arquivoS}\label{importauxe7uxe3o-dos-arquivos}}

\begin{Shaded}
\begin{Highlighting}[]
\NormalTok{discursos <-}\StringTok{ }\KeywordTok{read_csv}\NormalTok{(}\StringTok{'discursos.csv'}\NormalTok{) }\CommentTok{# "rodando" discurso no console, observa-se que ele já está no formato de um tibble.}
\end{Highlighting}
\end{Shaded}

\begin{verbatim}
## Parsed with column specification:
## cols(
##   .default = col_character(),
##   data_sessao = col_date(format = ""),
##   horario_inicio_discurso = col_time(format = ""),
##   horario_fim_discurso = col_time(format = ""),
##   duracao_discurso = col_double()
## )
\end{verbatim}

\begin{verbatim}
## See spec(...) for full column specifications.
\end{verbatim}

\begin{Shaded}
\begin{Highlighting}[]
\KeywordTok{head}\NormalTok{(discursos, }\DecValTok{10}\NormalTok{)}
\end{Highlighting}
\end{Shaded}

\begin{verbatim}
## # A tibble: 10 x 21
##    sequencial_sess~ data_sessao sigla_partido sigla_uf_partido bloco_parlament~
##    <chr>            <date>      <chr>         <chr>            <chr>           
##  1 01               2021-04-27  PSD           BA               <NA>            
##  2 01               2021-04-27  PP            PI               Bloco Parlament~
##  3 01               2021-04-27  PSD           BA               <NA>            
##  4 01               2021-04-27  PP            PI               Bloco Parlament~
##  5 01               2021-04-27  PSD           BA               <NA>            
##  6 01               2021-04-27  PP            PI               Bloco Parlament~
##  7 01               2021-04-27  PSD           BA               <NA>            
##  8 01               2021-04-27  PP            PI               Bloco Parlament~
##  9 01               2021-04-27  PSD           BA               <NA>            
## 10 01               2021-04-27  PP            PI               Bloco Parlament~
## # ... with 16 more variables: nome_discursante <chr>, genero_discursante <chr>,
## #   categoria_discursante <chr>, texto_discurso <chr>,
## #   horario_inicio_discurso <time>, horario_fim_discurso <time>,
## #   duracao_discurso <dbl>, sinalizacao_pela_ordem <chr>,
## #   sinalizacao_questao_ordem <chr>, sinalizacao_fora_microfone <chr>,
## #   sinalizacao_responder_questao_ordem <chr>,
## #   sinalizacao_por_videoconferencia <chr>, sinalizacao_para_interpelar <chr>,
## #   sinalizacao_para_expor <chr>, sinalizacao_para_depor <chr>,
## #   sinalizacao_como_presidente <chr>
\end{verbatim}

\hypertarget{categoria-dos-discursantes}{%
\subsection{Categoria dos
discursantes:}\label{categoria-dos-discursantes}}

Categoria dos discursantes (Senador/a):

\begin{Shaded}
\begin{Highlighting}[]
\NormalTok{discursos_senadores <-}\StringTok{ }\KeywordTok{filter}\NormalTok{(discursos, categoria_discursante }\OperatorTok{==}\StringTok{ 'Senador/a'}\NormalTok{)}\OperatorTok
\StringTok{        }\KeywordTok{select}\NormalTok{(nome_discursante, texto_discurso)}
\end{Highlighting}
\end{Shaded}

Categoria dos discursantes (Depoente/convidado):

\begin{Shaded}
\begin{Highlighting}[]
\NormalTok{discursos_depoentes <-}\StringTok{ }\KeywordTok{filter}\NormalTok{(discursos, categoria_discursante }\OperatorTok{==}\StringTok{ 'Depoente/Convidado'}\NormalTok{)}\OperatorTok
\StringTok{        }\KeywordTok{select}\NormalTok{(nome_discursante, texto_discurso)}
\end{Highlighting}
\end{Shaded}

\hypertarget{preprocessamento-padruxe3o-do-texto.}{%
\subsection{Preprocessamento padrão do
texto.}\label{preprocessamento-padruxe3o-do-texto.}}

\begin{Shaded}
\begin{Highlighting}[]
\CommentTok{# Senadores}
\NormalTok{discursos_senadores}\OperatorTok{$}\NormalTok{texto_discurso <-}\StringTok{ }\NormalTok{discursos_senadores}\OperatorTok{$}\NormalTok{texto_discurso }\OperatorTok
\StringTok{    }\KeywordTok{str_to_lower}\NormalTok{() }\OperatorTok\StringTok{                      }\CommentTok{# Caixa baixa.}
\StringTok{    }\KeywordTok{str_replace_all}\NormalTok{(}\StringTok{" *-+ *"}\NormalTok{, }\StringTok{""}\NormalTok{) }\OperatorTok\StringTok{       }\CommentTok{# Remove hífen.}
\StringTok{    }\KeywordTok{str_replace_all}\NormalTok{(}\StringTok{"[[:punct:]]"}\NormalTok{, }\StringTok{" "}\NormalTok{) }\OperatorTok\StringTok{ }\CommentTok{# Pontuação por espaço.}
\StringTok{    }\KeywordTok{removeNumbers}\NormalTok{() }\OperatorTok\StringTok{                     }\CommentTok{# Remove números.}
\StringTok{    }\KeywordTok{trimws}\NormalTok{()                                }\CommentTok{# Sem espaços nas bordas.}

\CommentTok{# Depoente/Convidado}
\NormalTok{discursos_depoentes}\OperatorTok{$}\NormalTok{texto_discurso <-}\StringTok{ }\NormalTok{discursos_depoentes}\OperatorTok{$}\NormalTok{texto_discurso }\OperatorTok
\StringTok{    }\KeywordTok{str_to_lower}\NormalTok{() }\OperatorTok\StringTok{                      }\CommentTok{# Caixa baixa.}
\StringTok{    }\KeywordTok{str_replace_all}\NormalTok{(}\StringTok{" *-+ *"}\NormalTok{, }\StringTok{""}\NormalTok{) }\OperatorTok\StringTok{       }\CommentTok{# Remove hífen.}
\StringTok{    }\KeywordTok{str_replace_all}\NormalTok{(}\StringTok{"[[:punct:]]"}\NormalTok{, }\StringTok{" "}\NormalTok{) }\OperatorTok\StringTok{ }\CommentTok{# Pontuação por espaço.}
\StringTok{    }\KeywordTok{removeNumbers}\NormalTok{() }\OperatorTok\StringTok{                     }\CommentTok{# Remove números.}
\StringTok{    }\KeywordTok{trimws}\NormalTok{()                                }\CommentTok{# Sem espaços nas bordas.}
\end{Highlighting}
\end{Shaded}

\hypertarget{remouxe7uxe3o-das-stop-words.}{%
\subsection{Remoção das stop
words.}\label{remouxe7uxe3o-das-stop-words.}}

\begin{Shaded}
\begin{Highlighting}[]
\NormalTok{discursos_senadores}\OperatorTok{$}\NormalTok{texto_discurso <-}\StringTok{ }\NormalTok{discursos_senadores}\OperatorTok{$}\NormalTok{texto_discurso }\OperatorTok
\StringTok{    }\KeywordTok{removeWords}\NormalTok{(}\DataTypeTok{words =} \KeywordTok{c}\NormalTok{(}\StringTok{"bom"}\NormalTok{, }\StringTok{"muito"}\NormalTok{, }\StringTok{"pouco"}\NormalTok{, }\StringTok{"obrigado"}\NormalTok{,}\StringTok{"obrigada"}\NormalTok{, }\StringTok{"dar"}\NormalTok{, }\StringTok{"sa"}\NormalTok{, }\StringTok{"favor"}\NormalTok{, }\StringTok{"roberto"}\NormalTok{, }\CommentTok{#removento obrigado e obrigada  }
                          \KeywordTok{stopwords}\NormalTok{(}\StringTok{'pt'}\NormalTok{)))}


\NormalTok{discursos_depoentes}\OperatorTok{$}\NormalTok{texto_discurso <-}\StringTok{ }\NormalTok{discursos_depoentes}\OperatorTok{$}\NormalTok{texto_discurso }\OperatorTok
\StringTok{    }\KeywordTok{removeWords}\NormalTok{(}\DataTypeTok{words =} \KeywordTok{c}\NormalTok{(}\StringTok{"bom"}\NormalTok{, }\StringTok{"muito"}\NormalTok{, }\StringTok{"pouco"}\NormalTok{, }\StringTok{"obrigado"}\NormalTok{,}\StringTok{"obrigada"}\NormalTok{, }\StringTok{"dar"}\NormalTok{, }\StringTok{"sa"}\NormalTok{, }\StringTok{"favor"}\NormalTok{, }\StringTok{"roberto"}\NormalTok{, }\CommentTok{#removento obrigado e obrigada  }
                          \KeywordTok{stopwords}\NormalTok{(}\StringTok{'pt'}\NormalTok{)))}
\end{Highlighting}
\end{Shaded}

\hypertarget{realizando-a-tokenizauxe7uxe3o-do-texto.}{%
\subsection{Realizando a ``tokenização'' do
texto.}\label{realizando-a-tokenizauxe7uxe3o-do-texto.}}

Para isso vamos carregar o package ``tidytext'':

\begin{Shaded}
\begin{Highlighting}[]
\CommentTok{# Senadores}
\NormalTok{discurso_senadores_unnested <-}\StringTok{ }\NormalTok{discursos_senadores }\OperatorTok
\StringTok{        }\KeywordTok{unnest_tokens}\NormalTok{(term, texto_discurso)}
        \CommentTok{# o primeiro argumento da fç unnest_tokens é o novo nome da coluna de saída que será criada, e o # segundo é a fonte, de onde será retirada. }
\CommentTok{# Notar que o número da linha de onde a palavra veio é mantido. }

\NormalTok{discurso_senadores_unnested}
\end{Highlighting}
\end{Shaded}

\begin{verbatim}
## # A tibble: 660,908 x 2
##    nome_discursante term      
##    <chr>            <chr>     
##  1 OTTO ALENCAR     invocando 
##  2 OTTO ALENCAR     proteção  
##  3 OTTO ALENCAR     deus      
##  4 OTTO ALENCAR     declaro   
##  5 OTTO ALENCAR     aberta    
##  6 OTTO ALENCAR     sessão    
##  7 OTTO ALENCAR     eleição   
##  8 OTTO ALENCAR     quórum    
##  9 OTTO ALENCAR     suficiente
## 10 OTTO ALENCAR     abertura  
## # ... with 660,898 more rows
\end{verbatim}

\begin{Shaded}
\begin{Highlighting}[]
\CommentTok{# Depoentes}
\NormalTok{discurso_depoentes_unnested <-}\StringTok{ }\NormalTok{discursos_depoentes }\OperatorTok
\StringTok{        }\KeywordTok{unnest_tokens}\NormalTok{(term, texto_discurso)}
        \CommentTok{# o primeiro argumento da fç unnest_tokens é o novo nome da coluna de saída que será criada, e o # segundo é a fonte, de onde será retirada. }
\CommentTok{# Notar que o número da linha de onde a palavra veio é mantido. }

\NormalTok{discurso_depoentes_unnested}
\end{Highlighting}
\end{Shaded}

\begin{verbatim}
## # A tibble: 315,547 x 2
##    nome_discursante       term       
##    <chr>                  <chr>      
##  1 LUIZ HENRIQUE MANDETTA dia        
##  2 LUIZ HENRIQUE MANDETTA todos      
##  3 LUIZ HENRIQUE MANDETTA prometo    
##  4 LUIZ HENRIQUE MANDETTA dia        
##  5 LUIZ HENRIQUE MANDETTA todos      
##  6 LUIZ HENRIQUE MANDETTA todas      
##  7 LUIZ HENRIQUE MANDETTA tempo      
##  8 LUIZ HENRIQUE MANDETTA cumprimento
##  9 LUIZ HENRIQUE MANDETTA presidente 
## 10 LUIZ HENRIQUE MANDETTA desta      
## # ... with 315,537 more rows
\end{verbatim}

\hypertarget{calculando-a-polaridade}{%
\subsection{Calculando a polaridade}\label{calculando-a-polaridade}}

\begin{Shaded}
\begin{Highlighting}[]
\CommentTok{# uma amostra do dicionario de termos rotulados}

\KeywordTok{sample_n}\NormalTok{(oplexicon_v3}\FloatTok{.0}\NormalTok{, }\DataTypeTok{size =} \DecValTok{20}\NormalTok{) }\OperatorTok
\StringTok{    }\KeywordTok{arrange}\NormalTok{(polarity)}
\end{Highlighting}
\end{Shaded}

\begin{verbatim}
##                term     type polarity polarity_revision
## 1          refutada      adj       -1                 A
## 2      #cuandoerani     htag       -1                 A
## 3         vingativo      adj       -1                 A
## 4             tenro      adj       -1                 A
## 5      interminavel      adj       -1                 M
## 6            selado      adj       -1                 A
## 7        dissipados      adj       -1                 A
## 8  fazer acusacao a vb n prp       -1                 A
## 9   individualistas      adj       -1                 A
## 10      delimitados      adj       -1                 A
## 11        esqualida      adj       -1                 M
## 12        seduzidos      adj       -1                 A
## 13         vergados      adj       -1                 A
## 14         perigoso      adj       -1                 M
## 15     conglomerada      adj        0                 A
## 16         botanico      adj        0                 M
## 17         desferir       vb        0                 A
## 18           sorver       vb        0                 A
## 19            viril      adj        1                 A
## 20           castos      adj        1                 A
\end{verbatim}

\begin{Shaded}
\begin{Highlighting}[]
\CommentTok{# Contagem por polaridade.}
\NormalTok{oplexicon_v3}\FloatTok{.0} \OperatorTok
\StringTok{    }\KeywordTok{count}\NormalTok{(polarity, }\DataTypeTok{sort =} \OtherTok{TRUE}\NormalTok{)}
\end{Highlighting}
\end{Shaded}

\begin{verbatim}
##   polarity     n
## 1       -1 14569
## 2        0  9002
## 3        1  8620
\end{verbatim}

\begin{Shaded}
\begin{Highlighting}[]
\CommentTok{# Contagem por classe gramatical.}
\NormalTok{oplexicon_v3}\FloatTok{.0} \OperatorTok
\StringTok{    }\KeywordTok{count}\NormalTok{(type, }\DataTypeTok{sort =} \OtherTok{TRUE}\NormalTok{)}
\end{Highlighting}
\end{Shaded}

\begin{verbatim}
##           type     n
## 1          adj 24475
## 2           vb  6889
## 3         htag   471
## 4 vb det n prp   103
## 5     vb n prp    91
## 6       vb adj    74
## 7         emot    66
## 8       vb adv    22
\end{verbatim}

\begin{Shaded}
\begin{Highlighting}[]
\CommentTok{# Faz o a junção por interseção.}
\NormalTok{tb_sen <-}\StringTok{ }\KeywordTok{inner_join}\NormalTok{(discurso_senadores_unnested,}
\NormalTok{                     oplexicon_v3}\FloatTok{.0}\NormalTok{[, }\KeywordTok{c}\NormalTok{(}\StringTok{"term"}\NormalTok{, }\StringTok{"polarity"}\NormalTok{)],}
                     \DataTypeTok{by =} \KeywordTok{c}\NormalTok{(}\StringTok{"term"}\NormalTok{ =}\StringTok{ "term"}\NormalTok{))}

\CommentTok{# Agora o termos tem sua polaridade presente na tabela.}
\KeywordTok{sample_n}\NormalTok{(tb_sen, }\DataTypeTok{size =} \DecValTok{20}\NormalTok{)}
\end{Highlighting}
\end{Shaded}

\begin{verbatim}
## # A tibble: 20 x 3
##    nome_discursante   term        polarity
##    <chr>              <chr>          <int>
##  1 RENAN CALHEIROS    dada               0
##  2 SIMONE TEBET       falar              0
##  3 OMAR AZIZ          responder          0
##  4 EDUARDO BRAGA      falta             -1
##  5 HUMBERTO COSTA     previsto           1
##  6 FLÁVIO BOLSONARO   pedir              1
##  7 EDUARDO GIRÃO      pedido             1
##  8 RENAN CALHEIROS    parlamentar        1
##  9 TASSO JEREISSATI   feitas             1
## 10 ROGÉRIO CARVALHO   aglomerar          0
## 11 HUMBERTO COSTA     ser                1
## 12 SIMONE TEBET       artista            0
## 13 RANDOLFE RODRIGUES importante         1
## 14 ZENAIDE MAIA       brasileiras        0
## 15 EDUARDO BRAGA      presidente         0
## 16 MARCOS ROGÉRIO     garantir           0
## 17 EDUARDO GIRÃO      utilizar           0
## 18 SIMONE TEBET       verdadeiro         1
## 19 RENAN CALHEIROS    citados            0
## 20 JEAN PAUL PRATES   agravante          1
\end{verbatim}

\begin{Shaded}
\begin{Highlighting}[]
\NormalTok{tb_depoente <-}\StringTok{ }\KeywordTok{inner_join}\NormalTok{(discurso_depoentes_unnested,}
\NormalTok{                     oplexicon_v3}\FloatTok{.0}\NormalTok{[, }\KeywordTok{c}\NormalTok{(}\StringTok{"term"}\NormalTok{, }\StringTok{"polarity"}\NormalTok{)],}
                     \DataTypeTok{by =} \KeywordTok{c}\NormalTok{(}\StringTok{"term"}\NormalTok{ =}\StringTok{ "term"}\NormalTok{))}

\CommentTok{# Agora o termos tem sua polaridade presente na tabela.}
\KeywordTok{sample_n}\NormalTok{(tb_depoente, }\DataTypeTok{size =} \DecValTok{20}\NormalTok{)}
\end{Highlighting}
\end{Shaded}

\begin{verbatim}
## # A tibble: 20 x 3
##    nome_discursante                            term         polarity
##    <chr>                                       <chr>           <int>
##  1 WILSON WITZEL                               destruir            0
##  2 FABIO WAJNGARTEN                            confrontar         -1
##  3 DIMAS TADEU COVAS                           assinado            0
##  4 JUREMA WERNECK                              potencial           1
##  5 FRANCISCO EDUARDO CARDOSO ALVES             residenciais        0
##  6 LUANA ARAÚJO                                testar              1
##  7 MARCELO QUEIROGA                            presidente          0
##  8 LUIZ HENRIQUE MANDETTA                      desfazer           -1
##  9 ANTONIO BARRA TORRES                        apontada           -1
## 10 OSMAR TERRA                                 querer              1
## 11 ANTONIO BARRA TORRES                        justificado         1
## 12 ERNESTO ARAÚJO                              improviso          -1
## 13 MARCELO QUEIROGA                            dizer               0
## 14 NISE HITOMI YAMAGUCHI                       passado            -1
## 15 WILSON WITZEL                               preciso             1
## 16 RICARDO ARIEL ZIMERMAN                      importante          1
## 17 MARCELO QUEIROGA                            importantes         1
## 18 LUIZ HENRIQUE MANDETTA                      reais               1
## 19 FRANCIELI FONTANA SUTILE TARDETTI FANTINATO estar               1
## 20 FRANCIELI FONTANA SUTILE TARDETTI FANTINATO descobertas         1
\end{verbatim}

\hypertarget{faz-a-agregauxe7uxe3o-da-polaridade-por-documento.}{%
\section{Faz a agregação da polaridade por
documento.}\label{faz-a-agregauxe7uxe3o-da-polaridade-por-documento.}}

\begin{Shaded}
\begin{Highlighting}[]
\CommentTok{# senadores}
\NormalTok{tb <-}\StringTok{ }\NormalTok{tb_sen }\OperatorTok
\StringTok{    }\KeywordTok{group_by}\NormalTok{(nome_discursante) }\OperatorTok
\StringTok{    }\KeywordTok{summarise}\NormalTok{(}\DataTypeTok{soma =} \KeywordTok{sum}\NormalTok{(polarity),}
              \DataTypeTok{n =} \KeywordTok{n}\NormalTok{(),}
              \DataTypeTok{sentiment =}\NormalTok{ soma}\OperatorTok{/}\NormalTok{n)}
\end{Highlighting}
\end{Shaded}

\begin{verbatim}
## `summarise()` ungrouping output (override with `.groups` argument)
\end{verbatim}

\begin{Shaded}
\begin{Highlighting}[]
\NormalTok{tb}
\end{Highlighting}
\end{Shaded}

\begin{verbatim}
## # A tibble: 42 x 4
##    nome_discursante         soma     n sentiment
##    <chr>                   <int> <int>     <dbl>
##  1 ALESSANDRO VIEIRA         689  3894    0.177 
##  2 ANGELO CORONEL             56   329    0.170 
##  3 CARLOS PORTINHO            27   139    0.194 
##  4 CIRO NOGUEIRA             144  1083    0.133 
##  5 EDUARDO BRAGA             294  2995    0.0982
##  6 EDUARDO GIRÃO            1112  6569    0.169 
##  7 ELIZIANE GAMA             580  4465    0.130 
##  8 FABIANO CONTARATO         538  3134    0.172 
##  9 FERNANDO BEZERRA COELHO   621  4718    0.132 
## 10 FLÁVIO BOLSONARO          252  1687    0.149 
## # ... with 32 more rows
\end{verbatim}

\begin{Shaded}
\begin{Highlighting}[]
\CommentTok{# depoentes}
\NormalTok{tb_ <-}\StringTok{ }\NormalTok{tb_depoente }\OperatorTok
\StringTok{    }\KeywordTok{group_by}\NormalTok{(nome_discursante) }\OperatorTok
\StringTok{    }\KeywordTok{summarise}\NormalTok{(}\DataTypeTok{soma =} \KeywordTok{sum}\NormalTok{(polarity),}
              \DataTypeTok{n =} \KeywordTok{n}\NormalTok{(),}
              \DataTypeTok{sentiment =}\NormalTok{ soma}\OperatorTok{/}\NormalTok{n)}
\end{Highlighting}
\end{Shaded}

\begin{verbatim}
## `summarise()` ungrouping output (override with `.groups` argument)
\end{verbatim}

\begin{Shaded}
\begin{Highlighting}[]
\NormalTok{tb_}
\end{Highlighting}
\end{Shaded}

\begin{verbatim}
## # A tibble: 43 x 4
##    nome_discursante                    soma     n sentiment
##    <chr>                              <int> <int>     <dbl>
##  1 ALBERTO ZACHARIAS TORON                6    57     0.105
##  2 ANTONIO BARRA TORRES                 499  2106     0.237
##  3 ANTÔNIO ELCIO FRANCO FILHO           259  1665     0.156
##  4 CARLOS MURILLO                       210   934     0.225
##  5 CARLOS ROBERTO WIZARD MARTINS         83   309     0.269
##  6 CLÁUDIO MAIEROVITCH                  196  1643     0.119
##  7 CRISTIANO ALBERTO HOSSRI CARVALHO    257  1286     0.200
##  8 DIMAS TADEU COVAS                    280  1656     0.169
##  9 EDUARDO PAZUELLO                     517  3641     0.142
## 10 EMANUELA BATISTA DE SOUZA MEDRADES   261  1237     0.211
## # ... with 33 more rows
\end{verbatim}

\hypertarget{desidade-expuxedrica-kernel-do-escore-de-sentimento.}{%
\section{Desidade expírica kernel do escore de
sentimento.}\label{desidade-expuxedrica-kernel-do-escore-de-sentimento.}}

\begin{Shaded}
\begin{Highlighting}[]
\KeywordTok{ggplot}\NormalTok{(tb, }\KeywordTok{aes}\NormalTok{(}\DataTypeTok{x =}\NormalTok{ sentiment)) }\OperatorTok{+}
\StringTok{    }\KeywordTok{geom_density}\NormalTok{(}\DataTypeTok{fill =} \StringTok{"orange"}\NormalTok{, }\DataTypeTok{alpha =} \FloatTok{0.25}\NormalTok{) }\OperatorTok{+}
\StringTok{    }\KeywordTok{geom_rug}\NormalTok{() }\OperatorTok{+}
\StringTok{    }\KeywordTok{labs}\NormalTok{(}\DataTypeTok{x =} \StringTok{"Polaridade"}\NormalTok{, }\DataTypeTok{y =} \StringTok{"Densidade"}\NormalTok{)}
\end{Highlighting}
\end{Shaded}

\includegraphics{script_estudo_AnaliseSentimentos_files/figure-latex/unnamed-chunk-9-1.pdf}

\begin{Shaded}
\begin{Highlighting}[]
\CommentTok{# depoentes}
\KeywordTok{ggplot}\NormalTok{(tb_, }\KeywordTok{aes}\NormalTok{(}\DataTypeTok{x =}\NormalTok{ sentiment)) }\OperatorTok{+}
\StringTok{    }\KeywordTok{geom_density}\NormalTok{(}\DataTypeTok{fill =} \StringTok{"orange"}\NormalTok{, }\DataTypeTok{alpha =} \FloatTok{0.25}\NormalTok{) }\OperatorTok{+}
\StringTok{    }\KeywordTok{geom_rug}\NormalTok{() }\OperatorTok{+}
\StringTok{    }\KeywordTok{labs}\NormalTok{(}\DataTypeTok{x =} \StringTok{"Polaridade"}\NormalTok{, }\DataTypeTok{y =} \StringTok{"Densidade"}\NormalTok{)}
\end{Highlighting}
\end{Shaded}

\includegraphics{script_estudo_AnaliseSentimentos_files/figure-latex/unnamed-chunk-9-2.pdf}

\hypertarget{frequuxeancia-relativa-acumulada.}{%
\section{Frequência relativa
acumulada.}\label{frequuxeancia-relativa-acumulada.}}

\begin{Shaded}
\begin{Highlighting}[]
\KeywordTok{ggplot}\NormalTok{(tb, }\KeywordTok{aes}\NormalTok{(}\DataTypeTok{x =}\NormalTok{ sentiment)) }\OperatorTok{+}
\StringTok{    }\KeywordTok{stat_ecdf}\NormalTok{() }\OperatorTok{+}
\StringTok{    }\KeywordTok{geom_rug}\NormalTok{() }\OperatorTok{+}
\StringTok{    }\KeywordTok{labs}\NormalTok{(}\DataTypeTok{x =} \StringTok{"Polaridade"}\NormalTok{, }\DataTypeTok{y =} \StringTok{"Frequência")}
\end{Highlighting}
\end{Shaded}

\includegraphics{script_estudo_AnaliseSentimentos_files/figure-latex/unnamed-chunk-10-1.pdf}

\begin{Shaded}
\begin{Highlighting}[]
\KeywordTok{ggplot}\NormalTok{(tb_, }\KeywordTok{aes}\NormalTok{(}\DataTypeTok{x =}\NormalTok{ sentiment)) }\OperatorTok{+}
\StringTok{    }\KeywordTok{stat_ecdf}\NormalTok{() }\OperatorTok{+}
\StringTok{    }\KeywordTok{geom_rug}\NormalTok{() }\OperatorTok{+}
\StringTok{    }\KeywordTok{labs}\NormalTok{(}\DataTypeTok{x =} \StringTok{"Polaridade"}\NormalTok{, }\DataTypeTok{y =} \StringTok{"Frequência")}
\end{Highlighting}
\end{Shaded}

\includegraphics{script_estudo_AnaliseSentimentos_files/figure-latex/unnamed-chunk-10-2.pdf}

\hypertarget{as-avaliauxe7uxf5es-mais-positivas.}{%
\section{As avaliações mais
positivas.}\label{as-avaliauxe7uxf5es-mais-positivas.}}

\begin{Shaded}
\begin{Highlighting}[]
\CommentTok{# Determina as frequências dos termos de polaridade não nula.}
\NormalTok{tb_words <-}\StringTok{ }\NormalTok{tb_sen }\OperatorTok
\StringTok{    }\KeywordTok{count}\NormalTok{(term, polarity, }\DataTypeTok{sort =} \OtherTok{TRUE}\NormalTok{) }\OperatorTok
\StringTok{    }\KeywordTok{filter}\NormalTok{(polarity }\OperatorTok{!=}\StringTok{ }\DecValTok{0}\NormalTok{)}

\NormalTok{tb_cloud <-}\StringTok{ }\NormalTok{tb_words }\OperatorTok
\StringTok{    }\KeywordTok{spread}\NormalTok{(}\DataTypeTok{key =} \StringTok{"polarity"}\NormalTok{, }\DataTypeTok{value =} \StringTok{"n"}\NormalTok{, }\DataTypeTok{fill =} \DecValTok{0}\NormalTok{) }\OperatorTok
\StringTok{    }\KeywordTok{rename}\NormalTok{(}\StringTok{"negative"}\NormalTok{ =}\StringTok{ "-1"}\NormalTok{, }\StringTok{"positive"}\NormalTok{ =}\StringTok{ "1"}\NormalTok{)}
\NormalTok{tb_cloud}
\end{Highlighting}
\end{Shaded}

\begin{verbatim}
## # A tibble: 3,435 x 3
##    term        negative positive
##    <chr>          <dbl>    <dbl>
##  1 abalados           1        0
##  2 abalizada          0        2
##  3 abalizadas         0        1
##  4 abandonada         1        0
##  5 abandonado         7        0
##  6 abandonados        2        0
##  7 abandonar          3        0
##  8 abastecer          0        4
##  9 abatido            1        0
## 10 abdicar            1        0
## # ... with 3,425 more rows
\end{verbatim}

\begin{Shaded}
\begin{Highlighting}[]
\NormalTok{tb <-}\StringTok{ }\KeywordTok{as.data.frame}\NormalTok{(tb_cloud[, }\KeywordTok{c}\NormalTok{(}\StringTok{"negative"}\NormalTok{, }\StringTok{"positive"}\NormalTok{)])}
\KeywordTok{rownames}\NormalTok{(tb) <-}\StringTok{ }\NormalTok{tb_cloud}\OperatorTok{$}\NormalTok{term}
\KeywordTok{head}\NormalTok{(tb)}
\end{Highlighting}
\end{Shaded}

\begin{verbatim}
##             negative positive
## abalados           1        0
## abalizada          0        2
## abalizadas         0        1
## abandonada         1        0
## abandonado         7        0
## abandonados        2        0
\end{verbatim}

\begin{Shaded}
\begin{Highlighting}[]
\KeywordTok{library}\NormalTok{(wordcloud)}
\end{Highlighting}
\end{Shaded}

\begin{verbatim}
## Warning: package 'wordcloud' was built under R version 4.0.5
\end{verbatim}

\begin{verbatim}
## Loading required package: RColorBrewer
\end{verbatim}

\begin{Shaded}
\begin{Highlighting}[]
\KeywordTok{comparison.cloud}\NormalTok{(tb,}
                 \DataTypeTok{colors =} \KeywordTok{c}\NormalTok{(}\StringTok{"#F8766D"}\NormalTok{, }\StringTok{"#00BFC4"}\NormalTok{),}
                 \DataTypeTok{max.words =} \KeywordTok{min}\NormalTok{(}\KeywordTok{nrow}\NormalTok{(tb), }\DecValTok{150}\NormalTok{))}
\end{Highlighting}
\end{Shaded}

\includegraphics{script_estudo_AnaliseSentimentos_files/figure-latex/unnamed-chunk-11-1.pdf}

\begin{Shaded}
\begin{Highlighting}[]
\CommentTok{# depoentes}

\NormalTok{tb_words_depoente <-}\StringTok{ }\NormalTok{tb_depoente }\OperatorTok
\StringTok{    }\KeywordTok{count}\NormalTok{(term, polarity, }\DataTypeTok{sort =} \OtherTok{TRUE}\NormalTok{) }\OperatorTok
\StringTok{    }\KeywordTok{filter}\NormalTok{(polarity }\OperatorTok{!=}\StringTok{ }\DecValTok{0}\NormalTok{)}

\NormalTok{tb_cloud <-}\StringTok{ }\NormalTok{tb_words_depoente }\OperatorTok
\StringTok{    }\KeywordTok{spread}\NormalTok{(}\DataTypeTok{key =} \StringTok{"polarity"}\NormalTok{, }\DataTypeTok{value =} \StringTok{"n"}\NormalTok{, }\DataTypeTok{fill =} \DecValTok{0}\NormalTok{) }\OperatorTok
\StringTok{    }\KeywordTok{rename}\NormalTok{(}\StringTok{"negative"}\NormalTok{ =}\StringTok{ "-1"}\NormalTok{, }\StringTok{"positive"}\NormalTok{ =}\StringTok{ "1"}\NormalTok{)}
\NormalTok{tb_cloud}
\end{Highlighting}
\end{Shaded}

\begin{verbatim}
## # A tibble: 2,600 x 3
##    term        negative positive
##    <chr>          <dbl>    <dbl>
##  1 abalada            1        0
##  2 abalado            2        0
##  3 abandonado         4        0
##  4 abandonados        1        0
##  5 abandonar          1        0
##  6 abarrotadas        1        0
##  7 abarrotado         1        0
##  8 abastada           0        1
##  9 abastecer          0        6
## 10 abater             1        0
## # ... with 2,590 more rows
\end{verbatim}

\begin{Shaded}
\begin{Highlighting}[]
\NormalTok{tb <-}\StringTok{ }\KeywordTok{as.data.frame}\NormalTok{(tb_cloud[, }\KeywordTok{c}\NormalTok{(}\StringTok{"negative"}\NormalTok{, }\StringTok{"positive"}\NormalTok{)])}
\KeywordTok{rownames}\NormalTok{(tb) <-}\StringTok{ }\NormalTok{tb_cloud}\OperatorTok{$}\NormalTok{term}
\KeywordTok{head}\NormalTok{(tb)}
\end{Highlighting}
\end{Shaded}

\begin{verbatim}
##             negative positive
## abalada            1        0
## abalado            2        0
## abandonado         4        0
## abandonados        1        0
## abandonar          1        0
## abarrotadas        1        0
\end{verbatim}

\begin{Shaded}
\begin{Highlighting}[]
\KeywordTok{library}\NormalTok{(wordcloud)}
\KeywordTok{comparison.cloud}\NormalTok{(tb,}
                 \DataTypeTok{colors =} \KeywordTok{c}\NormalTok{(}\StringTok{"#F8766D"}\NormalTok{, }\StringTok{"#00BFC4"}\NormalTok{),}
                 \DataTypeTok{max.words =} \KeywordTok{min}\NormalTok{(}\KeywordTok{nrow}\NormalTok{(tb), }\DecValTok{100}\NormalTok{))}
\end{Highlighting}
\end{Shaded}

\includegraphics{script_estudo_AnaliseSentimentos_files/figure-latex/unnamed-chunk-11-2.pdf}

\end{document}
